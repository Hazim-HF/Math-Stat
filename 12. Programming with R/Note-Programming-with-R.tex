% Options for packages loaded elsewhere
\PassOptionsToPackage{unicode}{hyperref}
\PassOptionsToPackage{hyphens}{url}
%
\documentclass[
]{article}
\usepackage{amsmath,amssymb}
\usepackage{iftex}
\ifPDFTeX
  \usepackage[T1]{fontenc}
  \usepackage[utf8]{inputenc}
  \usepackage{textcomp} % provide euro and other symbols
\else % if luatex or xetex
  \usepackage{unicode-math} % this also loads fontspec
  \defaultfontfeatures{Scale=MatchLowercase}
  \defaultfontfeatures[\rmfamily]{Ligatures=TeX,Scale=1}
\fi
\usepackage{lmodern}
\ifPDFTeX\else
  % xetex/luatex font selection
\fi
% Use upquote if available, for straight quotes in verbatim environments
\IfFileExists{upquote.sty}{\usepackage{upquote}}{}
\IfFileExists{microtype.sty}{% use microtype if available
  \usepackage[]{microtype}
  \UseMicrotypeSet[protrusion]{basicmath} % disable protrusion for tt fonts
}{}
\makeatletter
\@ifundefined{KOMAClassName}{% if non-KOMA class
  \IfFileExists{parskip.sty}{%
    \usepackage{parskip}
  }{% else
    \setlength{\parindent}{0pt}
    \setlength{\parskip}{6pt plus 2pt minus 1pt}}
}{% if KOMA class
  \KOMAoptions{parskip=half}}
\makeatother
\usepackage{xcolor}
\usepackage[margin=0.5cm]{geometry}
\usepackage{color}
\usepackage{fancyvrb}
\newcommand{\VerbBar}{|}
\newcommand{\VERB}{\Verb[commandchars=\\\{\}]}
\DefineVerbatimEnvironment{Highlighting}{Verbatim}{commandchars=\\\{\}}
% Add ',fontsize=\small' for more characters per line
\usepackage{framed}
\definecolor{shadecolor}{RGB}{248,248,248}
\newenvironment{Shaded}{\begin{snugshade}}{\end{snugshade}}
\newcommand{\AlertTok}[1]{\textcolor[rgb]{0.94,0.16,0.16}{#1}}
\newcommand{\AnnotationTok}[1]{\textcolor[rgb]{0.56,0.35,0.01}{\textbf{\textit{#1}}}}
\newcommand{\AttributeTok}[1]{\textcolor[rgb]{0.13,0.29,0.53}{#1}}
\newcommand{\BaseNTok}[1]{\textcolor[rgb]{0.00,0.00,0.81}{#1}}
\newcommand{\BuiltInTok}[1]{#1}
\newcommand{\CharTok}[1]{\textcolor[rgb]{0.31,0.60,0.02}{#1}}
\newcommand{\CommentTok}[1]{\textcolor[rgb]{0.56,0.35,0.01}{\textit{#1}}}
\newcommand{\CommentVarTok}[1]{\textcolor[rgb]{0.56,0.35,0.01}{\textbf{\textit{#1}}}}
\newcommand{\ConstantTok}[1]{\textcolor[rgb]{0.56,0.35,0.01}{#1}}
\newcommand{\ControlFlowTok}[1]{\textcolor[rgb]{0.13,0.29,0.53}{\textbf{#1}}}
\newcommand{\DataTypeTok}[1]{\textcolor[rgb]{0.13,0.29,0.53}{#1}}
\newcommand{\DecValTok}[1]{\textcolor[rgb]{0.00,0.00,0.81}{#1}}
\newcommand{\DocumentationTok}[1]{\textcolor[rgb]{0.56,0.35,0.01}{\textbf{\textit{#1}}}}
\newcommand{\ErrorTok}[1]{\textcolor[rgb]{0.64,0.00,0.00}{\textbf{#1}}}
\newcommand{\ExtensionTok}[1]{#1}
\newcommand{\FloatTok}[1]{\textcolor[rgb]{0.00,0.00,0.81}{#1}}
\newcommand{\FunctionTok}[1]{\textcolor[rgb]{0.13,0.29,0.53}{\textbf{#1}}}
\newcommand{\ImportTok}[1]{#1}
\newcommand{\InformationTok}[1]{\textcolor[rgb]{0.56,0.35,0.01}{\textbf{\textit{#1}}}}
\newcommand{\KeywordTok}[1]{\textcolor[rgb]{0.13,0.29,0.53}{\textbf{#1}}}
\newcommand{\NormalTok}[1]{#1}
\newcommand{\OperatorTok}[1]{\textcolor[rgb]{0.81,0.36,0.00}{\textbf{#1}}}
\newcommand{\OtherTok}[1]{\textcolor[rgb]{0.56,0.35,0.01}{#1}}
\newcommand{\PreprocessorTok}[1]{\textcolor[rgb]{0.56,0.35,0.01}{\textit{#1}}}
\newcommand{\RegionMarkerTok}[1]{#1}
\newcommand{\SpecialCharTok}[1]{\textcolor[rgb]{0.81,0.36,0.00}{\textbf{#1}}}
\newcommand{\SpecialStringTok}[1]{\textcolor[rgb]{0.31,0.60,0.02}{#1}}
\newcommand{\StringTok}[1]{\textcolor[rgb]{0.31,0.60,0.02}{#1}}
\newcommand{\VariableTok}[1]{\textcolor[rgb]{0.00,0.00,0.00}{#1}}
\newcommand{\VerbatimStringTok}[1]{\textcolor[rgb]{0.31,0.60,0.02}{#1}}
\newcommand{\WarningTok}[1]{\textcolor[rgb]{0.56,0.35,0.01}{\textbf{\textit{#1}}}}
\usepackage{graphicx}
\makeatletter
\def\maxwidth{\ifdim\Gin@nat@width>\linewidth\linewidth\else\Gin@nat@width\fi}
\def\maxheight{\ifdim\Gin@nat@height>\textheight\textheight\else\Gin@nat@height\fi}
\makeatother
% Scale images if necessary, so that they will not overflow the page
% margins by default, and it is still possible to overwrite the defaults
% using explicit options in \includegraphics[width, height, ...]{}
\setkeys{Gin}{width=\maxwidth,height=\maxheight,keepaspectratio}
% Set default figure placement to htbp
\makeatletter
\def\fps@figure{htbp}
\makeatother
\setlength{\emergencystretch}{3em} % prevent overfull lines
\providecommand{\tightlist}{%
  \setlength{\itemsep}{0pt}\setlength{\parskip}{0pt}}
\setcounter{secnumdepth}{-\maxdimen} % remove section numbering
\ifLuaTeX
  \usepackage{selnolig}  % disable illegal ligatures
\fi
\usepackage{bookmark}
\IfFileExists{xurl.sty}{\usepackage{xurl}}{} % add URL line breaks if available
\urlstyle{same}
\hypersetup{
  pdftitle={Programming with R},
  pdfauthor={Hazim Fitri},
  hidelinks,
  pdfcreator={LaTeX via pandoc}}

\title{Programming with R}
\author{Hazim Fitri}
\date{2025-01-19}

\begin{document}
\maketitle

{
\setcounter{tocdepth}{6}
\tableofcontents
}
\section{if else Statement}\label{if-else-statement}

\begin{Shaded}
\begin{Highlighting}[]
\NormalTok{x }\OtherTok{=} \DecValTok{0}
\ControlFlowTok{if}\NormalTok{ (x}\SpecialCharTok{\textgreater{}=}\DecValTok{0}\NormalTok{) \{}
  \FunctionTok{print}\NormalTok{(}\StringTok{\textquotesingle{}x is positive\textquotesingle{}}\NormalTok{)}
\NormalTok{\}}
\end{Highlighting}
\end{Shaded}

\begin{verbatim}
## [1] "x is positive"
\end{verbatim}

\begin{Shaded}
\begin{Highlighting}[]
\NormalTok{x }\OtherTok{=} \SpecialCharTok{{-}}\DecValTok{5}
\ControlFlowTok{if}\NormalTok{(x }\SpecialCharTok{\textgreater{}=} \DecValTok{0}\NormalTok{) \{}
  \FunctionTok{print}\NormalTok{(}\StringTok{\textquotesingle{}x is positive\textquotesingle{}}\NormalTok{)}
\NormalTok{\} }\ControlFlowTok{else}\NormalTok{ \{}
  \FunctionTok{print}\NormalTok{(}\StringTok{"x is negative"}\NormalTok{)}
\NormalTok{\}}
\end{Highlighting}
\end{Shaded}

\begin{verbatim}
## [1] "x is negative"
\end{verbatim}

\begin{Shaded}
\begin{Highlighting}[]
\NormalTok{x }\OtherTok{=} \DecValTok{0}
\ControlFlowTok{if}\NormalTok{ (x }\SpecialCharTok{\textgreater{}} \DecValTok{0}\NormalTok{) \{}
  \FunctionTok{print}\NormalTok{(}\StringTok{\textquotesingle{}x is positive\textquotesingle{}}\NormalTok{)}
\NormalTok{\} }\ControlFlowTok{else} \ControlFlowTok{if}\NormalTok{ (x }\SpecialCharTok{\textless{}} \DecValTok{0}\NormalTok{) \{}
  \FunctionTok{print}\NormalTok{(}\StringTok{\textquotesingle{}x is negative\textquotesingle{}}\NormalTok{)}
\NormalTok{\} }\ControlFlowTok{else} \ControlFlowTok{if}\NormalTok{ (x }\SpecialCharTok{==} \DecValTok{0}\NormalTok{)\{}
  \FunctionTok{print}\NormalTok{(}\StringTok{\textquotesingle{}x is zero\textquotesingle{}}\NormalTok{)}
\NormalTok{\}}
\end{Highlighting}
\end{Shaded}

\begin{verbatim}
## [1] "x is zero"
\end{verbatim}

\section{ifelse Function}\label{ifelse-function}

\begin{Shaded}
\begin{Highlighting}[]
\NormalTok{x }\OtherTok{=} \FunctionTok{c}\NormalTok{(}\SpecialCharTok{{-}}\DecValTok{3}\NormalTok{, }\DecValTok{0}\NormalTok{, }\DecValTok{3}\NormalTok{)}
\FunctionTok{ifelse}\NormalTok{(x }\SpecialCharTok{\textgreater{}=} \DecValTok{0}\NormalTok{, }\StringTok{\textquotesingle{}x is positive\textquotesingle{}}\NormalTok{, }\StringTok{\textquotesingle{}x is negative\textquotesingle{}}\NormalTok{)}
\end{Highlighting}
\end{Shaded}

\begin{verbatim}
## [1] "x is negative" "x is positive" "x is positive"
\end{verbatim}

\section{Switch Function}\label{switch-function}

\begin{Shaded}
\begin{Highlighting}[]
\NormalTok{x }\OtherTok{=} \DecValTok{3}
\ControlFlowTok{switch}\NormalTok{(x, }\StringTok{\textquotesingle{}red\textquotesingle{}}\NormalTok{, }\StringTok{\textquotesingle{}green\textquotesingle{}}\NormalTok{, }\StringTok{\textquotesingle{}blue\textquotesingle{}}\NormalTok{)}
\end{Highlighting}
\end{Shaded}

\begin{verbatim}
## [1] "blue"
\end{verbatim}

\section{Loop}\label{loop}

\subsection{For Loop}\label{for-loop}

\begin{Shaded}
\begin{Highlighting}[]
\NormalTok{x }\OtherTok{=} \DecValTok{1}
\ControlFlowTok{for}\NormalTok{ (i }\ControlFlowTok{in} \DecValTok{1}\SpecialCharTok{:}\DecValTok{3}\NormalTok{) \{}
\NormalTok{  x }\OtherTok{=}\NormalTok{ x }\SpecialCharTok{+} \DecValTok{1}
\NormalTok{\}}
\NormalTok{x}
\end{Highlighting}
\end{Shaded}

\begin{verbatim}
## [1] 4
\end{verbatim}

\begin{Shaded}
\begin{Highlighting}[]
\ControlFlowTok{for}\NormalTok{ (i }\ControlFlowTok{in} \DecValTok{1}\SpecialCharTok{:}\DecValTok{4}\NormalTok{) \{}
\NormalTok{  x }\OtherTok{=}\NormalTok{ x }\SpecialCharTok{+} \DecValTok{2}
\NormalTok{\}}
\NormalTok{x}
\end{Highlighting}
\end{Shaded}

\begin{verbatim}
## [1] 12
\end{verbatim}

\subsection{While Loop}\label{while-loop}

\begin{Shaded}
\begin{Highlighting}[]
\ControlFlowTok{while}\NormalTok{ (x }\SpecialCharTok{\textgreater{}} \DecValTok{0}\NormalTok{) \{}
\NormalTok{  x }\OtherTok{=}\NormalTok{ x }\SpecialCharTok{{-}} \DecValTok{5}
\NormalTok{\}}
\NormalTok{x}
\end{Highlighting}
\end{Shaded}

\begin{verbatim}
## [1] -3
\end{verbatim}

\subsection{Break Statement}\label{break-statement}

Break statement will break the loop

\begin{Shaded}
\begin{Highlighting}[]
\NormalTok{x }\OtherTok{=} \DecValTok{1}\SpecialCharTok{:}\DecValTok{5}
\ControlFlowTok{for}\NormalTok{ (y }\ControlFlowTok{in}\NormalTok{ x) \{}
  \ControlFlowTok{if}\NormalTok{ (y }\SpecialCharTok{==} \DecValTok{3}\NormalTok{) \{}
    \ControlFlowTok{break}
\NormalTok{  \}}
  \FunctionTok{print}\NormalTok{(y)}
\NormalTok{\}}
\end{Highlighting}
\end{Shaded}

\begin{verbatim}
## [1] 1
## [1] 2
\end{verbatim}

\subsection{Next Statement}\label{next-statement}

Skip current iteration without breaking the loop

\begin{Shaded}
\begin{Highlighting}[]
\NormalTok{x }\OtherTok{=} \DecValTok{1}\SpecialCharTok{:}\DecValTok{5}
\ControlFlowTok{for}\NormalTok{ (y }\ControlFlowTok{in}\NormalTok{ x) \{}
  \ControlFlowTok{if}\NormalTok{ (y }\SpecialCharTok{==} \DecValTok{3}\NormalTok{) \{}
    \ControlFlowTok{next}
\NormalTok{  \}}
  \FunctionTok{print}\NormalTok{(y)}
\NormalTok{\}}
\end{Highlighting}
\end{Shaded}

\begin{verbatim}
## [1] 1
## [1] 2
## [1] 4
## [1] 5
\end{verbatim}

\section{Function}\label{function}

local environment -\textgreater{} value that only exist within a
function

global environment -\textgreater{} value that only exist outside a
function

return() function -\textgreater{} to pass value into global environment

\begin{Shaded}
\begin{Highlighting}[]
\NormalTok{myfunction }\OtherTok{=} \ControlFlowTok{function}\NormalTok{ (x,y) \{}
\NormalTok{  a }\OtherTok{=}\NormalTok{ x }\SpecialCharTok{+}\NormalTok{ y}
  \FunctionTok{return}\NormalTok{(a)}
\NormalTok{\}}

\FunctionTok{myfunction}\NormalTok{(}\DecValTok{4}\NormalTok{,}\DecValTok{5}\NormalTok{)}
\end{Highlighting}
\end{Shaded}

\begin{verbatim}
## [1] 9
\end{verbatim}

\begin{Shaded}
\begin{Highlighting}[]
\NormalTok{square.it }\OtherTok{=} \ControlFlowTok{function}\NormalTok{ (x) \{}
\NormalTok{  b }\OtherTok{=}\NormalTok{ x }\SpecialCharTok{*}\NormalTok{ x}
  \FunctionTok{return}\NormalTok{(b)}
\NormalTok{\}}
\FunctionTok{square.it}\NormalTok{(}\DecValTok{8}\NormalTok{)}
\end{Highlighting}
\end{Shaded}

\begin{verbatim}
## [1] 64
\end{verbatim}

\begin{Shaded}
\begin{Highlighting}[]
\NormalTok{my.fun}\OtherTok{\textless{}{-}}\ControlFlowTok{function}\NormalTok{(X.matrix, y.vec, z.scalar)\{}
 \CommentTok{\#use previous function}
\NormalTok{ sq.scalar}\OtherTok{\textless{}{-}}\FunctionTok{square.it}\NormalTok{(z.scalar)}
\NormalTok{ mult}\OtherTok{\textless{}{-}}\NormalTok{X.matrix}\SpecialCharTok{\%*\%}\NormalTok{y.vec}
\NormalTok{ Final}\OtherTok{\textless{}{-}}\NormalTok{mult}\SpecialCharTok{*}\NormalTok{sq.scalar}
 \FunctionTok{return}\NormalTok{(Final)}
\NormalTok{ \}}
\NormalTok{ my.mat}\OtherTok{\textless{}{-}}\FunctionTok{cbind}\NormalTok{(}\FunctionTok{c}\NormalTok{(}\DecValTok{1}\NormalTok{,}\DecValTok{2}\NormalTok{,}\DecValTok{3}\NormalTok{),}\FunctionTok{c}\NormalTok{(}\DecValTok{3}\NormalTok{,}\DecValTok{4}\NormalTok{,}\DecValTok{5}\NormalTok{))}
\NormalTok{ my.vec}\OtherTok{\textless{}{-}}\FunctionTok{c}\NormalTok{(}\DecValTok{5}\NormalTok{,}\DecValTok{6}\NormalTok{)}
 \FunctionTok{my.fun}\NormalTok{(}\AttributeTok{X.matrix=}\NormalTok{my.mat, }\AttributeTok{y.vec=}\NormalTok{my.vec, }\AttributeTok{z.scalar=}\DecValTok{9}\NormalTok{)}
\end{Highlighting}
\end{Shaded}

\begin{verbatim}
##      [,1]
## [1,] 1863
## [2,] 2754
## [3,] 3645
\end{verbatim}

\section{Exercise}\label{exercise}

\subsection{Write a function that computes the coefficient of variation
and plot the histogram of a
vector}\label{write-a-function-that-computes-the-coefficient-of-variation-and-plot-the-histogram-of-a-vector}

\begin{Shaded}
\begin{Highlighting}[]
\NormalTok{mycoef }\OtherTok{=} \ControlFlowTok{function}\NormalTok{ (x) \{}
\NormalTok{  z }\OtherTok{=} \FunctionTok{sd}\NormalTok{(x) }\SpecialCharTok{/} \FunctionTok{mean}\NormalTok{(x) }\SpecialCharTok{*} \DecValTok{100}
  \FunctionTok{hist}\NormalTok{(x)}
  \FunctionTok{return}\NormalTok{(z)}
\NormalTok{\}}
\NormalTok{j }\OtherTok{=} \FunctionTok{c}\NormalTok{(}\DecValTok{1}\NormalTok{,}\DecValTok{2}\NormalTok{,}\DecValTok{3}\NormalTok{,}\DecValTok{4}\NormalTok{,}\DecValTok{5}\NormalTok{,}\DecValTok{6}\NormalTok{,}\DecValTok{7}\NormalTok{,}\DecValTok{8}\NormalTok{,}\DecValTok{9}\NormalTok{)}
\FunctionTok{mycoef}\NormalTok{(j)}
\end{Highlighting}
\end{Shaded}

\includegraphics{Note-Programming-with-R_files/figure-latex/unnamed-chunk-3-1.pdf}

\begin{verbatim}
## [1] 54.77226
\end{verbatim}

\begin{Shaded}
\begin{Highlighting}[]
\CommentTok{\# Function to compute the coefficient of variation and plot histogram}
\NormalTok{cv\_and\_histogram }\OtherTok{\textless{}{-}} \ControlFlowTok{function}\NormalTok{(data) \{}
  \CommentTok{\# Calculate the coefficient of variation}
\NormalTok{  cv }\OtherTok{\textless{}{-}} \FunctionTok{sd}\NormalTok{(data) }\SpecialCharTok{/} \FunctionTok{mean}\NormalTok{(data)}
  
  \CommentTok{\# Plot the histogram}
  \FunctionTok{hist}\NormalTok{(data, }\AttributeTok{main =} \StringTok{"Histogram of Data"}\NormalTok{, }\AttributeTok{xlab =} \StringTok{"Values"}\NormalTok{, }\AttributeTok{col =} \StringTok{"lightblue"}\NormalTok{, }\AttributeTok{border =} \StringTok{"black"}\NormalTok{)}
  
  \CommentTok{\# Return the coefficient of variation}
  \FunctionTok{return}\NormalTok{(cv)}
\NormalTok{\}}

\CommentTok{\# Example usage}
\NormalTok{data\_vector }\OtherTok{\textless{}{-}} \FunctionTok{c}\NormalTok{(}\DecValTok{1}\NormalTok{, }\DecValTok{2}\NormalTok{, }\DecValTok{3}\NormalTok{, }\DecValTok{4}\NormalTok{, }\DecValTok{5}\NormalTok{, }\DecValTok{6}\NormalTok{, }\DecValTok{7}\NormalTok{, }\DecValTok{8}\NormalTok{, }\DecValTok{9}\NormalTok{, }\DecValTok{10}\NormalTok{)}

\NormalTok{cv\_value }\OtherTok{\textless{}{-}} \FunctionTok{cv\_and\_histogram}\NormalTok{(data\_vector)}
\end{Highlighting}
\end{Shaded}

\includegraphics{Note-Programming-with-R_files/figure-latex/blackbox_answer1-1.pdf}

\begin{Shaded}
\begin{Highlighting}[]
\FunctionTok{print}\NormalTok{(}\FunctionTok{paste}\NormalTok{(}\StringTok{"Coefficient of Variation:"}\NormalTok{, cv\_value))}
\end{Highlighting}
\end{Shaded}

\begin{verbatim}
## [1] "Coefficient of Variation: 0.55048188256318"
\end{verbatim}

\subsection{Suppose a researcher obtained the coordinates of n points
and wish to find the maximum distance between two points from all the
different pairs of points possible, produce an R function which helps to
find the maximum distance between the coordinates of one particular
point, (𝑥𝑖, 𝑦𝑖), with the coordinates of the other (n--1) points. The
distance, d, between coordinates of points i and j, can be calculated as
follows:}\label{suppose-a-researcher-obtained-the-coordinates-of-n-points-and-wish-to-find-the-maximum-distance-between-two-points-from-all-the-different-pairs-of-points-possible-produce-an-r-function-which-helps-to-find-the-maximum-distance-between-the-coordinates-of-one-particular-point-ux1d465ux1d456-ux1d466ux1d456-with-the-coordinates-of-the-other-n1-points.-the-distance-d-between-coordinates-of-points-i-and-j-can-be-calculated-as-follows}

\begin{Shaded}
\begin{Highlighting}[]
\NormalTok{distance }\OtherTok{=} \ControlFlowTok{function}\NormalTok{ (x1, y1, x2, y2) \{}
\NormalTok{  d }\OtherTok{=} \FunctionTok{sqrt}\NormalTok{(((x1}\SpecialCharTok{{-}}\NormalTok{x2) }\SpecialCharTok{*}\NormalTok{ (x1}\SpecialCharTok{{-}}\NormalTok{x2)) }\SpecialCharTok{+}\NormalTok{ ((y1}\SpecialCharTok{{-}}\NormalTok{y2) }\SpecialCharTok{*}\NormalTok{ (y1}\SpecialCharTok{{-}}\NormalTok{y2)))}
  \FunctionTok{return}\NormalTok{(d)}
\NormalTok{\}}
\FunctionTok{distance}\NormalTok{(}\DecValTok{3}\NormalTok{,}\DecValTok{4}\NormalTok{,}\DecValTok{5}\NormalTok{,}\DecValTok{6}\NormalTok{)}
\end{Highlighting}
\end{Shaded}

\begin{verbatim}
## [1] 2.828427
\end{verbatim}

\begin{Shaded}
\begin{Highlighting}[]
\CommentTok{\# Function to calculate the maximum distance}
\NormalTok{max\_distance }\OtherTok{\textless{}{-}} \ControlFlowTok{function}\NormalTok{(points, point\_index) \{}
\NormalTok{  n }\OtherTok{\textless{}{-}} \FunctionTok{nrow}\NormalTok{(points)}
\NormalTok{  distances }\OtherTok{\textless{}{-}} \FunctionTok{numeric}\NormalTok{(n }\SpecialCharTok{{-}} \DecValTok{1}\NormalTok{)}
  
  \CommentTok{\# Get the coordinates of the reference point}
\NormalTok{  ref\_point }\OtherTok{\textless{}{-}}\NormalTok{ points[point\_index, ]}
  
  \CommentTok{\# Loop through each point to calculate distance}
\NormalTok{  count }\OtherTok{\textless{}{-}} \DecValTok{1}
  \ControlFlowTok{for}\NormalTok{ (i }\ControlFlowTok{in} \DecValTok{1}\SpecialCharTok{:}\NormalTok{n) \{}
    \ControlFlowTok{if}\NormalTok{ (i }\SpecialCharTok{!=}\NormalTok{ point\_index) \{}
\NormalTok{      x\_diff }\OtherTok{\textless{}{-}}\NormalTok{ points[i, }\DecValTok{1}\NormalTok{] }\SpecialCharTok{{-}}\NormalTok{ ref\_point[}\DecValTok{1}\NormalTok{]}
\NormalTok{      y\_diff }\OtherTok{\textless{}{-}}\NormalTok{ points[i, }\DecValTok{2}\NormalTok{] }\SpecialCharTok{{-}}\NormalTok{ ref\_point[}\DecValTok{2}\NormalTok{]}
\NormalTok{      distances[count] }\OtherTok{\textless{}{-}} \FunctionTok{sqrt}\NormalTok{(x\_diff}\SpecialCharTok{\^{}}\DecValTok{2} \SpecialCharTok{+}\NormalTok{ y\_diff}\SpecialCharTok{\^{}}\DecValTok{2}\NormalTok{)}
\NormalTok{      count }\OtherTok{\textless{}{-}}\NormalTok{ count }\SpecialCharTok{+} \DecValTok{1}
\NormalTok{    \}}
\NormalTok{  \}}
  
  \CommentTok{\# Return the maximum distance}
  \FunctionTok{return}\NormalTok{(}\FunctionTok{max}\NormalTok{(distances))}
\NormalTok{\}}

\CommentTok{\# Example usage}
\NormalTok{points }\OtherTok{\textless{}{-}} \FunctionTok{matrix}\NormalTok{(}\FunctionTok{c}\NormalTok{(}\DecValTok{1}\NormalTok{, }\DecValTok{2}\NormalTok{,}
                   \DecValTok{3}\NormalTok{, }\DecValTok{4}\NormalTok{,}
                   \DecValTok{5}\NormalTok{, }\DecValTok{6}\NormalTok{,}
                   \DecValTok{7}\NormalTok{, }\DecValTok{8}\NormalTok{), }\AttributeTok{byrow =} \ConstantTok{TRUE}\NormalTok{, }\AttributeTok{ncol =} \DecValTok{2}\NormalTok{)}

\NormalTok{point\_index }\OtherTok{\textless{}{-}} \DecValTok{2}  \CommentTok{\# Example, getting the max distance for the second point}

\NormalTok{max\_dist }\OtherTok{\textless{}{-}} \FunctionTok{max\_distance}\NormalTok{(points, point\_index)}

\FunctionTok{print}\NormalTok{(}\FunctionTok{paste}\NormalTok{(}\StringTok{"Maximum Distance:"}\NormalTok{, max\_dist))}
\end{Highlighting}
\end{Shaded}

\begin{verbatim}
## [1] "Maximum Distance: 5.65685424949238"
\end{verbatim}

\subsection{Mann-Kendall test is a statistical test used to determine
the existence of monotonic trend in a data set
X}\label{mann-kendall-test-is-a-statistical-test-used-to-determine-the-existence-of-monotonic-trend-in-a-data-set-x}

\subsubsection{Write an R function that computes the test statistic,
SMK, of the Mann-Kendall test for a data set X of size n which is given
as
follows:}\label{write-an-r-function-that-computes-the-test-statistic-smk-of-the-mann-kendall-test-for-a-data-set-x-of-size-n-which-is-given-as-follows}

\begin{Shaded}
\begin{Highlighting}[]
\NormalTok{x }\OtherTok{=} \FunctionTok{c}\NormalTok{(}\DecValTok{1}\NormalTok{,}\DecValTok{2}\NormalTok{,}\DecValTok{3}\NormalTok{,}\DecValTok{4}\NormalTok{,}\SpecialCharTok{{-}}\DecValTok{9}\NormalTok{,}\DecValTok{0}\NormalTok{)}
\FunctionTok{sign}\NormalTok{(x)}
\end{Highlighting}
\end{Shaded}

\begin{verbatim}
## [1]  1  1  1  1 -1  0
\end{verbatim}

\begin{Shaded}
\begin{Highlighting}[]
\CommentTok{\# Function to compute the Mann{-}Kendall test statistic}
\NormalTok{mann\_kendall\_statistic }\OtherTok{\textless{}{-}} \ControlFlowTok{function}\NormalTok{(X) \{}
\NormalTok{  n }\OtherTok{\textless{}{-}} \FunctionTok{length}\NormalTok{(X)  }\CommentTok{\# Size of the dataset}
\NormalTok{  S }\OtherTok{\textless{}{-}} \DecValTok{0}          \CommentTok{\# Initialize the test statistic}
  
  \CommentTok{\# Loop through all pairs of observations}
  \ControlFlowTok{for}\NormalTok{ (i }\ControlFlowTok{in} \DecValTok{1}\SpecialCharTok{:}\NormalTok{(n }\SpecialCharTok{{-}} \DecValTok{1}\NormalTok{)) \{}
    \ControlFlowTok{for}\NormalTok{ (j }\ControlFlowTok{in}\NormalTok{ (i }\SpecialCharTok{+} \DecValTok{1}\NormalTok{)}\SpecialCharTok{:}\NormalTok{n) \{}
      \ControlFlowTok{if}\NormalTok{ (X[j] }\SpecialCharTok{\textgreater{}}\NormalTok{ X[i]) \{}
\NormalTok{        S }\OtherTok{\textless{}{-}}\NormalTok{ S }\SpecialCharTok{+} \DecValTok{1}  \CommentTok{\# Count the number of increasing pairs}
\NormalTok{      \} }\ControlFlowTok{else} \ControlFlowTok{if}\NormalTok{ (X[j] }\SpecialCharTok{\textless{}}\NormalTok{ X[i]) \{}
\NormalTok{        S }\OtherTok{\textless{}{-}}\NormalTok{ S }\SpecialCharTok{{-}} \DecValTok{1}  \CommentTok{\# Count the number of decreasing pairs}
\NormalTok{      \}}
\NormalTok{    \}}
\NormalTok{  \}}
  
  \FunctionTok{return}\NormalTok{(S)  }\CommentTok{\# Return the test statistic}
\NormalTok{\}}

\CommentTok{\# Example usage}
\NormalTok{data }\OtherTok{\textless{}{-}} \FunctionTok{c}\NormalTok{(}\DecValTok{3}\NormalTok{, }\DecValTok{1}\NormalTok{, }\DecValTok{4}\NormalTok{, }\DecValTok{1}\NormalTok{, }\DecValTok{5}\NormalTok{, }\DecValTok{9}\NormalTok{, }\DecValTok{2}\NormalTok{, }\DecValTok{6}\NormalTok{)}
\NormalTok{S }\OtherTok{\textless{}{-}} \FunctionTok{mann\_kendall\_statistic}\NormalTok{(data)}
\FunctionTok{print}\NormalTok{(S)}
\end{Highlighting}
\end{Shaded}

\begin{verbatim}
## [1] 11
\end{verbatim}

\subsubsection{If X is a series with no tie values, then the
standardized test statistic, ZMK, is written
as}\label{if-x-is-a-series-with-no-tie-values-then-the-standardized-test-statistic-zmk-is-written-as}

\begin{Shaded}
\begin{Highlighting}[]
\CommentTok{\# Function to compute the Mann{-}Kendall test statistic and standardized statistic, ZMK}
\NormalTok{mann\_kendall\_test }\OtherTok{\textless{}{-}} \ControlFlowTok{function}\NormalTok{(X) \{}
\NormalTok{  n }\OtherTok{\textless{}{-}} \FunctionTok{length}\NormalTok{(X)}
\NormalTok{  S }\OtherTok{\textless{}{-}} \DecValTok{0}

  \CommentTok{\# Loop to calculate the test statistic}
  \ControlFlowTok{for}\NormalTok{ (i }\ControlFlowTok{in} \DecValTok{1}\SpecialCharTok{:}\NormalTok{(n }\SpecialCharTok{{-}} \DecValTok{1}\NormalTok{)) \{}
    \ControlFlowTok{for}\NormalTok{ (j }\ControlFlowTok{in}\NormalTok{ (i }\SpecialCharTok{+} \DecValTok{1}\NormalTok{)}\SpecialCharTok{:}\NormalTok{n) \{}
\NormalTok{      S }\OtherTok{\textless{}{-}}\NormalTok{ S }\SpecialCharTok{+} \FunctionTok{sign}\NormalTok{(X[j] }\SpecialCharTok{{-}}\NormalTok{ X[i])}
\NormalTok{    \}}
\NormalTok{  \}}
  
  \CommentTok{\# Adjusting the variance of S}
\NormalTok{  var\_S }\OtherTok{\textless{}{-}}\NormalTok{ (n }\SpecialCharTok{*}\NormalTok{ (n }\SpecialCharTok{{-}} \DecValTok{1}\NormalTok{) }\SpecialCharTok{*}\NormalTok{ (}\DecValTok{2} \SpecialCharTok{*}\NormalTok{ n }\SpecialCharTok{+} \DecValTok{5}\NormalTok{)) }\SpecialCharTok{/} \DecValTok{18}
  
  \CommentTok{\# Computing the standardized test statistic, ZMK}
  \ControlFlowTok{if}\NormalTok{ (S }\SpecialCharTok{\textgreater{}} \DecValTok{0}\NormalTok{) \{}
\NormalTok{    ZMK }\OtherTok{\textless{}{-}}\NormalTok{ (S }\SpecialCharTok{{-}} \DecValTok{1}\NormalTok{) }\SpecialCharTok{/} \FunctionTok{sqrt}\NormalTok{(var\_S)}
\NormalTok{  \} }\ControlFlowTok{else} \ControlFlowTok{if}\NormalTok{ (S }\SpecialCharTok{\textless{}} \DecValTok{0}\NormalTok{) \{}
\NormalTok{    ZMK }\OtherTok{\textless{}{-}}\NormalTok{ (S }\SpecialCharTok{+} \DecValTok{1}\NormalTok{) }\SpecialCharTok{/} \FunctionTok{sqrt}\NormalTok{(var\_S)}
\NormalTok{  \} }\ControlFlowTok{else}\NormalTok{ \{}
\NormalTok{    ZMK }\OtherTok{\textless{}{-}} \DecValTok{0}
\NormalTok{  \}}
  
  \FunctionTok{return}\NormalTok{(}\FunctionTok{list}\NormalTok{(}\AttributeTok{SMK =}\NormalTok{ S, }\AttributeTok{ZMK =}\NormalTok{ ZMK))}
\NormalTok{\}}

\CommentTok{\# Example usage}
\NormalTok{data\_set\_X }\OtherTok{\textless{}{-}} \FunctionTok{c}\NormalTok{(}\DecValTok{3}\NormalTok{, }\DecValTok{2}\NormalTok{, }\DecValTok{5}\NormalTok{, }\DecValTok{4}\NormalTok{, }\DecValTok{6}\NormalTok{, }\DecValTok{7}\NormalTok{, }\DecValTok{2}\NormalTok{, }\DecValTok{8}\NormalTok{, }\DecValTok{9}\NormalTok{, }\DecValTok{1}\NormalTok{)}

\NormalTok{result }\OtherTok{\textless{}{-}} \FunctionTok{mann\_kendall\_test}\NormalTok{(data\_set\_X)}

\FunctionTok{print}\NormalTok{(}\FunctionTok{paste}\NormalTok{(}\StringTok{"Mann{-}Kendall Test Statistic (SMK):"}\NormalTok{, result}\SpecialCharTok{$}\NormalTok{SMK))}
\end{Highlighting}
\end{Shaded}

\begin{verbatim}
## [1] "Mann-Kendall Test Statistic (SMK): 12"
\end{verbatim}

\begin{Shaded}
\begin{Highlighting}[]
\FunctionTok{print}\NormalTok{(}\FunctionTok{paste}\NormalTok{(}\StringTok{"Standardized Test Statistic (ZMK):"}\NormalTok{, result}\SpecialCharTok{$}\NormalTok{ZMK))}
\end{Highlighting}
\end{Shaded}

\begin{verbatim}
## [1] "Standardized Test Statistic (ZMK): 0.983869910099907"
\end{verbatim}

\end{document}
